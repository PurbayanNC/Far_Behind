% \date{\today}
\date{}
\title{\vspace{-1cm}Codebook- Team Far\_Behind\\IIT Delhi, India}
% Compiled and edited by Brian Bi
\documentclass[12pt]{extarticle}
\setlength{\parindent}{0.0in}
\usepackage{mathptmx}
\usepackage{amsmath}
\usepackage{multicol}
\usepackage[landscape,a4paper,twoside=false,top=4mm,bottom=4mm,left=15mm,right=4mm]{geometry}
\pagestyle{myheadings}
\markright{}
\usepackage{listings}
\usepackage{color}
\renewcommand{\baselinestretch}{0.7} 

\usepackage{titlesec}
\titlespacing*{\section}
{0pt}{0.0ex plus 0ex minus .0ex}{0.0ex plus .0ex}
\titlespacing*{\subsection}
{0pt}{0.0ex plus 0ex minus .0ex}{0.0ex plus .0ex}
\titlespacing*{\subsubsection}
{0pt}{0.0ex plus 0ex minus .0ex}{0.0ex plus .0ex}


\lstset{
	tabsize=2,
	basicstyle=\fontsize{12}{5.5}\ttfamily,
	% basicstyle=\ttfamily\scriptsize,
	%upquote=true,
	aboveskip={0.00\baselineskip},
	columns=fixed,
	showstringspaces=false,
	extendedchars=true,
	breaklines=true,
	prebreak = \raisebox{0ex}[0ex][0ex]{\ensuremath{\hookleftarrow}},
	frame=single,
	rulecolor=\color[rgb]{0.75,0.75,0.75},
	showtabs=false,
	showspaces=false,
	showstringspaces=false,
	keywordstyle=\color[rgb]{0,0,1},
	commentstyle=\color[rgb]{0.133,0.545,0.133},
	stringstyle=\color[rgb]{0.627,0.126,0.941},
%	literate={\ \ }{{\ }}1,
}

\author{
	Ayush Ranjan, Naman Jain, Manish Tanwar
}

\begin{document}
\maketitle
		\begin{multicols*}{2}
			
			\setlength{\parskip}{0.0in}
%			\tableofcontents
			\setlength{\parskip}{0.0in}
			
			\textbf{2. Data Structures : }
			\newline
			2.1 -  Fenwick,\quad 2.2 - 2D -BIT, \quad2.3 - Segment Tree , \quad2.4 - Persistent Segment Tree, \quad2.5 - DP Optimization
			\newline
			\textbf{3. Flows and Matching : }
			\newline
			3. -1  General Matching, \quad3.2 - Global Mincut, \quad3.3 - Hopcroft Matching, \quad3.4 - Dinic, \quad3.5 - Ford Fulkerson, \quad3.6 - MCMF, \quad3.7 - MinCost Matching
			\newline
			\textbf{4. Geometry : }
			\newline
			4.1 -  Geometry , \quad4.2 - Convex Hull, \quad4.3 - Li Chao Tree, \quad4.4 - Convex Hull Trick
						\newline
			\textbf{5. Trees : }
			\newline
			5.1 -  BlockCut Tree , \quad5.2 - Bridge Tree , \quad5.3 - Dominator Tree, \quad5.4 - Bridges Online, \quad5.5 - HLD , \quad5.6 - LCA , \quad5.7 - Centroid Decompostion
			\newline
			
			\textbf{6. Maths : }
			\newline
			6.1 -  Chinese Remainder Theorem, \quad6.2 - Discrete Log, \quad6.3 - NTT, \quad6.4 - Online FFT , \quad6.5 - Langrange Interpolation, \quad6.6 - Matrix Struct, \quad6.7 - nCr(Non Prime Modulo), \quad6.8 - Primitive Root Generator, \quad6.9 - Math Miscellaneous, \quad6.10 -  Group Theory \quad6.11 -  Gaussian Elimination \quad6.12 - Inclusion-Exclusion
			
			\textbf{7. Strings}
			\newline
			7.1 - Hashing Theory, \quad7.2 - Manacher, \quad7.3 - Trie, \quad7.4 - Z -algorithm, \quad7.5 - Aho Corasick, \quad7.6 - KMP, \quad7.7 - Palindrome Tree, \quad7.8 - Suffix Array, \quad7.9 - Suffix Tree, \quad7.10 - Suffix Automaton
			\newline
			
			\subsubsection*{Ideas} 
			\textbf{Analysis Complexity Carefully,\quad Relate To Theory},\quad First Find Solution for small sub-problems, \quad
			Div and Conq\quad, Brute force and observe,\quad (+1,-1),\quad Dominator Tree for Directed Graph,\quad use fenwick,\quad Monte Carlo,\quad Summation Interchange,\quad Clever Optimization of brute force(binary search/ignore),\quad Try solving problem backwards
			\newline
			
			\subsubsection*{Tree Ideas}
			a. LCA \hspace{0.5cm}
			b. DSU on trees (possibly make greater depth as heavy child)\hspace{0.5cm}
			c. Centroid Decomposition\hspace{0.5cm}
			d. HLD\hspace{0.5cm}
			e. Euler Tour (dfs order/bfs order/any other ordering)\hspace{0.5cm}
			f. Pass some structure(set/array/..) in dfs (segment/fenwick)\hspace{0.5cm}
			g. Reachability Tree (construction using dsu) \hspace{0.5cm}
			h. Dominator Tree (directed graphs) \hspace{0.5cm}
			i. Biconnectivity \hspace{0.5cm}
			j. DFS tree \hspace{0.5cm}
			
			\input contents.tex			
			\subsubsection*{M\"obius Function} % Brian Bi
			$\mu(n) = \begin{cases}
			0 & \text{$n$ not squarefree} \\
			1 & \text{$n$ squarefree w/ even no. of prime factors} \\
			-1 & \text{$n$ squarefree w/ odd no. of prime factors} \\
			\end{cases}$
			\par
			Note that $\mu(a) \mu(b) = \mu(ab)$ for $a, b$ relatively prime
			\par
			Also $\sum_{d \mid n} \mu(d) = \begin{cases} 1 & \text{if $n = 1$} \\
			0 & \text{otherwise} \end{cases}$
			
			\textbf{M\"obius Inversion}
			If $g(n) = \sum_{d|n} f(d)$ for all $n \ge 1$, then
			$f(n) = \sum_{d|n} \mu(d)g(n/d)$ for all $n \ge 1$.
			
			\par\vskip 7pt
			\subsubsection*{Burnside's Lemma (Text)} % Wesley May and Brian Bi
			The number of orbits of a set $X$ under the group action $G$ equals the average
			number of elements of $X$ fixed by the elements of $G$.
			
			Here's an example. Consider a square of $2n$ times $2n$ cells. How many ways
			are there to color it into $X$ colors, up to rotations and/or reflections?
			Here, the group has only 8 elements (rotations by 0, 90, 180 and 270 degrees,
			reflections over two diagonals, over a vertical line and over a horizontal
			line). Every coloring stays itself after rotating by 0 degrees, so that
			rotation has $X^{4n^2}$ fixed points. Rotation by 180 degrees and reflections
			over a horizonal/vertical line split all cells in pairs that must be of the
			same color for a coloring to be unaffected by such rotation/reflection, thus
			there exist $X^{2n^2}$ such colorings for each of them. Rotations by 90 and 270
			degrees split cells in groups of four, thus yielding $X^{n^2}$ fixed colorings.
			Reflections over diagonals split cells into $2n$ groups of 1 (the diagonal
			itself) and $2n^2-n$ groups of 2 (all remaining cells), thus yielding
			$X^{2n^2-n+2n}=X^{2n^2+n}$ unaffected colorings.  So, the answer is
			$(X^{4n^2}+3X^{2n^2}+2X^{n^2}+2X^{2n^2+n})/8$.
			Every tree with $n$ vertices has $n-1$ edges.
			\par\vskip 2pt
			\subsubsection*{Trees-Kraft inequality:}
			If the depths of the leaves of a binary tree are $d_1 \ldots d_n$:
			$
			\sum_{i=1}^n 2^{- d_i} \leq 1,
			$
			and equality holds only if every internal node has 2 sons.
			\par\vskip 2pt
			\subsubsection*{Euler Tour:}
			\begin{itemize}
			\itemsep0em
			\item Undirected graph iff All vertices have even degree, all non-zero degree vertices are in a single connected component.(Can be decomposed into edge-disjoint cycles)
			\item Directed graph iff For each vertex in-degree=out-degree, all non-zero degree vertices are in a single strongly connected component.(Decomposible into directed-edge disjoint cycles)
			\end{itemize}
			\subsubsection*{Euler Trail:}
			\begin{itemize}
			\itemsep0em 
			\item Undirdected graph iff exactly 0 or 2 vertices have odd degree, single connected component(consider zero degree vertices only).
			\item Directed graph iff at most one vertex has (out-degree)-(in-degree) = 1, at most one vertex has (in-degree)-(out-degree) = 1, every other vertex has equal in-degree and out-degree, and all of its vertices with nonzero degree belong to a single connected component of the underlying undirected graph.
			\end{itemize}

			\par\vskip 3pt
			\textbf{Master method:}
				$$T(n) = aT(n/b) + f(n), \quad a\geq 1, b > 1$$
			If $\exists \epsilon > 0$ such that $f(n) = O(n^{\log_b a - \epsilon})$ then $T(n) = \Theta(n^{\log_b a}).$
			\par
			If $f(n) = \Theta(n^{\log_b a})$ then
			$T(n) = \Theta(n^{\log_b a} \log_2 n).$
			\newline If $\exists \epsilon > 0$ such that $f(n) = \Omega(n^{\log_b a + \epsilon})$,
			and $\exists c < 1$ such that $a f(n/b) \leq cf(n)$ for large $n$,
			then $T(n) = \Theta(f(n)).$
			\par\vskip 3pt
			\subsubsection*{Probability:}
			Variance, standard deviation:
			$Var[X] = E[X^2] - E[X]^2$
			\par\vskip 2pt
			\hspace{1cm} Poisson distribution: \hspace{2cm} Normal (Gaussian) distribution:
			$$
			\Pr[X = k] = {e^{-\lambda} \lambda^k \over k!}, \quad  E[X] = \lambda \quad \bigg| \quad p(x) = {1 \over \sqrt{2 \pi} \sigma} e^{-(x-\mu)^2/2\sigma^2}, \quad E[X] = \mu.
			$$
			The ``coupon collector'':
			We are given a random coupon each day,
			and there are $n$ different types of coupons.
			The distribution of coupons is uniform.
			The expected number of days to pass before we to collect all $n$ types is $n H_n.$
			\par\vskip 2pt
			\subsubsection*{Miscellaneous:}
			\begin{enumerate}
				\itemsep0em 
				\item 			Radius of inscribed circle for Right Angle Tringle:
				$\quad {A B \over A + B + C}$
				\item Law of cosine: 			$c^2 = a^2 + b^2 - 2ab \cos C$
				\item Area of a triangle: Area:
				$A = \frac{1}{2} h c
					= \frac{1}{2} a b \sin C
					= {c^2 \sin A \sin B \over 2 \sin C}.$
				\item 			$\det A = \sum_\pi \prod_{i=1}^n \text{sign}(\pi) a_{i,\pi(i)}$, 
				for Permanents remove sign.
				\item Perfect Numbers: $x$ is an even perfect number iff $x = 2^{n-1}(2^n - 1)$ and $2^n - 1$ is prime.
				\item Wilson's theorem: $n$ is a prime iff
				$(n-1)! \equiv -1 \bmod n.$
				\item If graph $G$ is planar then $n - m + f = 2$, so
				$f \leq 2n - 4, \quad m \leq 3 n - 6.$
				Any planar graph has a vertex with degree $\leq 5$.
				\item Dirichlet power series:
				$A(x) =\sum_{i=1}^\infty{a_i \over i^x}$
				\item Coefficient of $x^r$ in $(1-x)^{-n}$ is ${n+r-1}\choose{r}$.
				\item Stirling numbers (1st kind): 
				\newline $s(n,2) = (n-1)!H_{n-1}$, \quad
				$s(n,k) = (n-1) s({n-1},k) + s({n-1},{k-1})$
				\item Stirling numbers (2nd kind): \quad $ S(n,k) = k S(n-1,k) + S(n-1,k-1)$
	\newline $ S(n,k) = \frac{1}{k!}\sum\limits_{j=0}^{k}(-1)^{k-j}\binom{k}{j}j^n $, \quad $S(n,m) = \sum_k {n \choose k} S({k+1},{m+1})(-1)^{n-k}$
				\item Catalan Numbers: Binary trees with $n+1$ vertices \quad $C_n = {1\over n+1}{2n \choose n}$
			\end{enumerate}
		\subsubsection*{For Bipartite Graphs}
		\begin{enumerate}
			\itemsep0em 
			\item Min-edge cover($me$) = Max-independent set($mi$) (G has no isolated vertex).
			\item Min-vertex cover($mv$) = Max matching($mm$) \hspace{5mm} $mi + mv = |V|$,\hspace{5mm} $mi \geq \frac{|V|}{2}$
			\item Min-edge cover subgraph is a combination of star graphs.
			\item Min Vertex cover : In residual graph of max flow pick all the vertices in 
			L(left bipartite) not reachable from s(whose edges are cut) and lly in R reachable from s.
			\item Min-edge cover(no isolated vertex) : Find max matching, take all those edges, for vertices not covered
			take any edge.
		\end{enumerate}
		
			$$(1) \quad \sum_{i=1}^n i^m = {1 \over m+1} \bigg[ (n+1)^{m+1} - 1 - \sum_{i=1}^n \big((i+1)^{m+1} - i^{m+1} - (m+1)i^m\big) \bigg]$$
		
			$$ (2) \quad {x \over (1 - x)^2} = \sum_{i=0}^\infty i x^i, \hspace{5mm}
			(3) \quad \ln (1 + x) = \sum_{i=1}^\infty (-1)^{i+1} {x^i \over i}, \hspace{5mm} (4) \quad {x \over 1 - x - x^2} = \sum_{i=0}^\infty F_i x^i$$
			
			$$ (5) \quad (x_1 + x_2 + ... + x_k)^n = \sum_{c_1 + c_2 + ... + c_k = n}
			\frac{n!}{c_1!...c_k!} x_1^{c_1} x_2^{c_2} ... x_k^{c_k},
			(6) \quad {1 \over (1-x)^{n+1}}
				= \sum_{i=0}^\infty {i+ n \choose i} x^i$$
%			$$ (6) \quad {1 \over (1-x)^{n+1}}
%			= \sum_{i=0}^\infty {i+ n \choose i} x^i $$
		
	
	$$(1) \quad \int \tanh x \, dx = \ln \vert \cosh x \vert, \hspace{3mm}
	(2) \quad \int \coth x \, dx = \ln \vert \sinh x \vert,
	(3) \quad \int {dx  \over a^2 - x^2} = {1 \over 2a} \ln\left\vert{a + x \over a - x}\right\vert, $$
	$$ (4) \quad \int {dx  \over \sqrt{a^2 + x^2}}= \ln \left(x + \sqrt{a^2 + x^2}\right) \hspace{0.2cm} (a > 0),(5) \quad \int {dx  \over a^2 + x^2}= \frac 1 a \arctan \frac x a \quad (a > 0), $$
	$$(6) \quad \int \sqrt{a^2 - x^2} \, dx = \frac x 2 \sqrt{a^2 - x^2} + \frac{a^2}{2} \arcsin \frac x a \quad (a > 0) \quad (7) \quad \int {dx  \over \sqrt{a^2 - x^2}} = \arcsin \frac x a \quad (a > 0) $$
	
		\subsubsection*{Fibonacci:}
	\begin{enumerate}
		\itemsep0em 
		\item $F_{-i} = (-1)^{i-1} F_i$,  $ \quad F_i = \frac{1}{\sqrt{5}} \left(\phi^i - \hat{\phi}^i\right)$
		\item Cassini's identity: $F_{i+1} F_{i-1} - F^2_i = (-1)^i$ \quad \hbox{for $i > 0$,}
		\item Additive Rule:
		\hspace{2mm}
		$F_{n+k} = F_k F_{n+1} + F_{k-1} F_n,$ \hspace{2mm}
		$F_{2n} = F_n F_{n+1} + F_{n-1} F_n$
		
		\item Every integer $n$ has a unique representation
		$ n = F_{k_1} + F_{k_2} + \cdots + F_{k_m},$
		where $k_i \geq k_{i+1} + 2$ for $1 \leq i < m$ and $k_m \geq 2$.
	\end{enumerate}
	\subsubsection*{Primes}
	$\forall (a,b)$, The largest prime smaller than $10^a$ is $p = 10^{a} - b$
	\newline 
	\par
	$(1,3), \hspace{1mm}(2,3), \hspace{1mm}(3,3), \hspace{1mm}(4,27), \hspace{1mm}(5,9), \hspace{1mm}(6,17), \hspace{1mm}(7,9), \hspace{1mm}(8,11), \hspace{1mm}(9,63), \hspace{1mm}(10,33),$
		\newline 
	\par
	$ \hspace{1mm}(11,23), \hspace{1mm}(12,11), \hspace{1mm}(13,29), \hspace{1mm}(14,27), \hspace{1mm}(15,11), \hspace{1mm}(16,63), \hspace{1mm}(17,3), \hspace{1mm}(18,11)$
	\par\vskip 3pt
	
	
	%\subsubsection*{C++ Sublime Build Extra}
	%\raggedbottom\lstinputlisting[language=c++]{code/Syntax_Related/%extra_build.cpp}
	\vfill

		\end{multicols*}
\end{document}
