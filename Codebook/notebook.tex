% \date{\today}
\date{}
\title{\vspace{-1cm}Codebook- Team Far\_Behind\\IIT Delhi, India}
% Compiled and edited by Brian Bi
\documentclass[12pt]{extarticle}
\setlength{\parindent}{0.0in}
\usepackage{mathptmx}
\usepackage{amsmath}
\usepackage{multicol}
\usepackage[landscape,a4paper,twoside=false,top=4mm,bottom=4mm,left=15mm,right=4mm]{geometry}
\pagestyle{myheadings}
\markright{}
\usepackage{listings}
\usepackage{color}

\usepackage{titlesec}
\titlespacing*{\section}
{0pt}{0.0ex plus 0ex minus .0ex}{0.0ex plus .0ex}
\titlespacing*{\subsection}
{0pt}{0.0ex plus 0ex minus .0ex}{0.0ex plus .0ex}
\titlespacing*{\subsubsection}
{0pt}{0.0ex plus 0ex minus .0ex}{0.0ex plus .0ex}

\lstset{
	tabsize=2,
	basicstyle=\fontsize{12}{0}\ttfamily,
	% basicstyle=\ttfamily\scriptsize,
	%upquote=true,
	aboveskip={0.00\baselineskip},
	columns=fixed,
	showstringspaces=false,
	extendedchars=true,
	breaklines=true,
	prebreak = \raisebox{0ex}[0ex][0ex]{\ensuremath{\hookleftarrow}},
	frame=single,
	rulecolor=\color[rgb]{0.75,0.75,0.75},
	showtabs=false,
	showspaces=false,
	showstringspaces=false,
	keywordstyle=\color[rgb]{0,0,1},
	commentstyle=\color[rgb]{0.133,0.545,0.133},
	stringstyle=\color[rgb]{0.627,0.126,0.941},
%	literate={\ \ }{{\ }}1,
}

\author{
	Ayush Ranjan, Naman Jain, Manish Tanwar
}

\begin{document}
\maketitle
		\begin{multicols*}{2}
			
			\setlength{\parskip}{0.0in}
			\tableofcontents
			\setlength{\parskip}{0.0in}
			
			\input contents.tex			
			\subsubsection*{M\"obius Function} % Brian Bi
			$\mu(n) = \begin{cases}
			0 & \text{$n$ not squarefree} \\
			1 & \text{$n$ squarefree w/ even no. of prime factors} \\
			-1 & \text{$n$ squarefree w/ odd no. of prime factors} \\
			\end{cases}$
			\par
			Note that $\mu(a) \mu(b) = \mu(ab)$ for $a, b$ relatively prime
			\par
			Also $\sum_{d \mid n} \mu(d) = \begin{cases} 1 & \text{if $n = 1$} \\
			0 & \text{otherwise} \end{cases}$
			
			\textbf{M\"obius Inversion}
			If $g(n) = \sum_{d|n} f(d)$ for all $n \ge 1$, then
			$f(n) = \sum_{d|n} \mu(d)g(n/d)$ for all $n \ge 1$.
			
			
			\subsubsection*{Burnside's Lemma (Text)} % Wesley May and Brian Bi
			The number of orbits of a set $X$ under the group action $G$ equals the average
			number of elements of $X$ fixed by the elements of $G$.
			
			Here's an example. Consider a square of $2n$ times $2n$ cells. How many ways
			are there to color it into $X$ colors, up to rotations and/or reflections?
			Here, the group has only 8 elements (rotations by 0, 90, 180 and 270 degrees,
			reflections over two diagonals, over a vertical line and over a horizontal
			line). Every coloring stays itself after rotating by 0 degrees, so that
			rotation has $X^{4n^2}$ fixed points. Rotation by 180 degrees and reflections
			over a horizonal/vertical line split all cells in pairs that must be of the
			same color for a coloring to be unaffected by such rotation/reflection, thus
			there exist $X^{2n^2}$ such colorings for each of them. Rotations by 90 and 270
			degrees split cells in groups of four, thus yielding $X^{n^2}$ fixed colorings.
			Reflections over diagonals split cells into $2n$ groups of 1 (the diagonal
			itself) and $2n^2-n$ groups of 2 (all remaining cells), thus yielding
			$X^{2n^2-n+2n}=X^{2n^2+n}$ unaffected colorings.  So, the answer is
			$(X^{4n^2}+3X^{2n^2}+2X^{n^2}+2X^{2n^2+n})/8$.
			Every tree with $n$ vertices has $n-1$ edges.
			
			\subsubsection*{Trees-Kraft inequality:}
			If the depths of the leaves of a binary tree are $d_1 \ldots d_n$:
			$
			\sum_{i=1}^n 2^{- d_i} \leq 1,
			$
			and equality holds only if every internal node has 2 sons.

			\par\vskip 3pt
			\textbf{Master method:}
			$$T(n) = aT(n/b) + f(n), \quad a\geq 1, b > 1$$
			If $\exists \epsilon > 0$ such that $f(n) = O(n^{\log_b a - \epsilon})$ then
			$$T(n) = \Theta(n^{\log_b a}).$$
			\par
			If $f(n) = \Theta(n^{\log_b a})$ then
			$$T(n) = \Theta(n^{\log_b a} \log_2 n).$$
			If $\exists \epsilon > 0$ such that $f(n) = \Omega(n^{\log_b a + \epsilon})$,
			and $\exists c < 1$ such that $a f(n/b) \leq cf(n)$ for large $n$,
			then
			$$T(n) = \Theta(f(n)).$$
			\subsubsection*{Probability:}
			Variance, standard deviation:
			$Var[X] = E[X^2] - E[X]^2$
			\par\vskip 5pt
			Poisson distribution:
			$$
			\Pr[X = k] = {e^{-\lambda} \lambda^k \over k!}, \quad  E[X] = \lambda.
			$$
			Normal (Gaussian) distribution:
			$$
			p(x) = {1 \over \sqrt{2 \pi} \sigma} e^{-(x-\mu)^2/2\sigma^2}, \quad E[X] = \mu.
			$$
			The ``coupon collector'':
			We are given a random coupon each day,
			and there are $n$ different types of coupons.
			The distribution of coupons is uniform.
			The expected number of days to pass before we to collect all $n$ types is $n H_n.$
			\subsubsection*{Miscellaneous:}
			
			\begin{enumerate}
				\item 			Radius of inscribed circle for Right Angle Tringle:
				$\quad {A B \over A + B + C}$
				\item Law of cosine: 			$c^2 = a^2 + b^2 - 2ab \cos C$
				\item Area of a triangle: Area:
				$A = \frac{1}{2} h c
					= \frac{1}{2} a b \sin C
					= {c^2 \sin A \sin B \over 2 \sin C}.$
				\item 			$\det A = \sum_\pi \prod_{i=1}^n \text{sign}(\pi) a_{i,\pi(i)}$, 
				for Permanents remove sign.
				\item Perfect Numbers: $x$ is an even perfect number iff $x = 2^{n-1}(2^n - 1)$ and $2^n - 1$ is prime.
				\item Wilson's theorem: $n$ is a prime iff
				$(n-1)! \equiv -1 \bmod n.$
				\item If graph $G$ is planar then $n - m + f = 2$, so
				$f \leq 2n - 4, \quad m \leq 3 n - 6.$
				Any planar graph has a vertex with degree $\leq 5$.
				\item Dirichlet power series:
				$A(x) =\sum_{i=1}^\infty{a_i \over i^x}$
				
			\end{enumerate}
			$$ {x \over (1 - x)^2} = \sum_{i=0}^\infty i x^i, \hspace{5mm}
			\ln (1 + x) = \sum_{i=1}^\infty (-1)^{i+1} {x^i \over i}, \hspace{5mm} {x \over 1 - x - x^2} = \sum_{i=0}^\infty F_i x^i$$
			$$(x_1 + x_2 + ... + x_k)^n = \sum_{c_1 + c_2 + ... + c_k = n}
			\frac{n!}{c_1! c_2! ... c_k!} x_1^{c_1} x_2^{c_2} ... x_k^{c_k}$$
			
		
		\subsubsection*{Fibonacci:}
		\begin{enumerate}
		\item $F_{-i} = (-1)^{i-1} F_i$,  $ \quad F_i = \frac{1}{\sqrt{5}} \left(\phi^i - \hat{\phi}^i\right)$
		\item Cassini's identity: $F_{i+1} F_{i-1} - F^2_i = (-1)^i$ \quad \hbox{for $i > 0$,}
		\item Addictive Rule:
		\hspace{2mm}
		$F_{n+k} = F_k F_{n+1} + F_{k-1} F_n,$ \hspace{2mm}
		$F_{2n} = F_n F_{n+1} + F_{n-1} F_n$
		
		\item Every integer $n$ has a unique representation
		$ n = F_{k_1} + F_{k_2} + \cdots + F_{k_m},$
		where $k_i \geq k_{i+1} + 2$ for $1 \leq i < m$ and $k_m \geq 2$.
		\end{enumerate}
		
		\end{multicols*}
\end{document}
